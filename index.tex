% Options for packages loaded elsewhere
\PassOptionsToPackage{unicode}{hyperref}
\PassOptionsToPackage{hyphens}{url}
%
\documentclass[
]{article}
\usepackage{amsmath,amssymb}
\usepackage{lmodern}
\usepackage{iftex}
\ifPDFTeX
  \usepackage[T1]{fontenc}
  \usepackage[utf8]{inputenc}
  \usepackage{textcomp} % provide euro and other symbols
\else % if luatex or xetex
  \usepackage{unicode-math}
  \defaultfontfeatures{Scale=MatchLowercase}
  \defaultfontfeatures[\rmfamily]{Ligatures=TeX,Scale=1}
\fi
% Use upquote if available, for straight quotes in verbatim environments
\IfFileExists{upquote.sty}{\usepackage{upquote}}{}
\IfFileExists{microtype.sty}{% use microtype if available
  \usepackage[]{microtype}
  \UseMicrotypeSet[protrusion]{basicmath} % disable protrusion for tt fonts
}{}
\makeatletter
\@ifundefined{KOMAClassName}{% if non-KOMA class
  \IfFileExists{parskip.sty}{%
    \usepackage{parskip}
  }{% else
    \setlength{\parindent}{0pt}
    \setlength{\parskip}{6pt plus 2pt minus 1pt}}
}{% if KOMA class
  \KOMAoptions{parskip=half}}
\makeatother
\usepackage{xcolor}
\usepackage[margin=1in]{geometry}
\usepackage{graphicx}
\makeatletter
\def\maxwidth{\ifdim\Gin@nat@width>\linewidth\linewidth\else\Gin@nat@width\fi}
\def\maxheight{\ifdim\Gin@nat@height>\textheight\textheight\else\Gin@nat@height\fi}
\makeatother
% Scale images if necessary, so that they will not overflow the page
% margins by default, and it is still possible to overwrite the defaults
% using explicit options in \includegraphics[width, height, ...]{}
\setkeys{Gin}{width=\maxwidth,height=\maxheight,keepaspectratio}
% Set default figure placement to htbp
\makeatletter
\def\fps@figure{htbp}
\makeatother
\setlength{\emergencystretch}{3em} % prevent overfull lines
\providecommand{\tightlist}{%
  \setlength{\itemsep}{0pt}\setlength{\parskip}{0pt}}
\setcounter{secnumdepth}{-\maxdimen} % remove section numbering
\ifLuaTeX
  \usepackage{selnolig}  % disable illegal ligatures
\fi
\IfFileExists{bookmark.sty}{\usepackage{bookmark}}{\usepackage{hyperref}}
\IfFileExists{xurl.sty}{\usepackage{xurl}}{} % add URL line breaks if available
\urlstyle{same} % disable monospaced font for URLs
\hypersetup{
  pdftitle={Exploratory Analysis},
  hidelinks,
  pdfcreator={LaTeX via pandoc}}

\title{Exploratory Analysis}
\author{}
\date{\vspace{-2.5em}}

\begin{document}
\maketitle

mentalhealth\_df \textless-
read.csv(``\url{https://raw.githubusercontent.com/info-201a-wi23/exploratory-analysis-BadaLee2000/main/depression_anxiety_data.csv}'',
stringsAsFactors = FALSE) mentalhealth\_df

\hypertarget{project-title}{%
\subsection{Project title}\label{project-title}}

Do Gender, Class Standing, and Age affect students' mental health?

\hypertarget{authors}{%
\subsubsection{Authors}\label{authors}}

Jason Cabusao, Jamie Kim, Bada Lee, and Tamia Ouch

\hypertarget{date}{%
\subsubsection{Date}\label{date}}

February 20, 2023 (Winter 2023)

\hypertarget{abstract}{%
\subsubsection{Abstract}\label{abstract}}

Our main question is if students' mental health has been affected in a
negative way based on the factors including age, gender, and class
standing. This question is important because depression and anxiety are
very common among students thesedays which affect their studies and
social well-being. Accordingly, we plan to compare each factor with
depression severity, depression diagnosis, depressiveness, suicidal,
anxiety severity, anxiousness, and anxiety diagnosis.

\hypertarget{keywords}{%
\subsubsection{Keywords}\label{keywords}}

Keywords: Mental Health; Student Life; Depression

\hypertarget{introduction}{%
\subsubsection{Introduction}\label{introduction}}

Briefly introduce your project. Include 3-5 research questions. What
motivates the questions? Why are they important? (at least 200 words)

\begin{enumerate}
\def\labelenumi{\arabic{enumi})}
\tightlist
\item
  Does gender affect students' mental health?
\item
  Does age affect students' mental health?
\item
  Does class standing affect students' mental health?
\item
  Which age group struggle the most ``depression severity and ONE OTHER
  COLUMN (NEEDS TO BE EDITED)?
\item
  Which gender has more depression severity?
\end{enumerate}

The motivation behind the research questions is the success of students
in both well-being and academic performances. With the variables that
make students different such as age, gender, and class standing, it is
possible to find out in what age, in which gender, and in which school
year, the students get the most depression and anxiety. More
specifically, the factors, age, gender, and class standing, will be
compared with depression severity, depression diagnosis, and anxiety
diagnosis. For example, if there is a specific age or school year or
gender that most students get stressed, depressed, or anxious, it is a
great idea for their parents, teachers, siblings to take care of the
students more attentively since the students' acquaintances and the
environment influence them a lot.

It is really important for students to have good mental health because
it affects how we think, feel, and act. Mental health can also affect
their education, social life, and emotional well-being. It is also
important for students to realize that mental health is not something to
be ashamed of. To produce better academic performances as students, it
is essential to keep their mental health strong and stable. Looking into
the data set will give us an idea of how many students feel the lots of
pressure to put into academics over mental health. Hopefully, we find
out what groups struggle the most and find a way to offer help
virtually.

\hypertarget{related-work}{%
\subsubsection{Related Work}\label{related-work}}

Describe your topic and related work in this space. You must include 3
citations to related work (URLs to similar work, high quality articles
from the popular press, research papers, etc. ) Please use a standard
citation style of your choice. (at least 200 words)

Our topic is about students' mental health. We want to look at gender,
class standing, age to see if there is a determining or peak factor when
it comes to low mental health. This can help us provide accurate
assistance and reassurance. There have been many studies done
surrounding this topic. Some include:

This piece of related work looks into the mental health of students
based on race.

Sarah Lipson, Sasha Zhou, Sara Abelson, Justin Heinze \& Matthew Jirsa
(2022) Trends in college student mental health and help-seeking by
race/ethnicity, journal of Affective Disorders,
\url{https://www.sciencedirect.com/science/article/pii/S0165032722002774}

This piece of related work looks into how student debt can be connected
to mental health.

Richard Cooke, Michael Barkham, Kerry Audin, Margaret Bradley \& John
Davy (2004) Student debt and its relation to student mental health,
Journal of Further and Higher Education, 28:1, 53-66, DOI:
10.1080/0309877032000161814

This piece of related work looks at mental health and its relation to
relocation during covid-19.

Rachel Conrad, Hyeouk Hahm, Amanda Koire, Stephanie Pinder-Amaker, \&
Cindy Liu (2021) College student mental health risks during the COVID-19
pandemic: Implications of campus relocation, Journal of Psychiatric
Research,
\url{https://www.sciencedirect.com/science/article/pii/S0022395621000650}

\hypertarget{the-dataset}{%
\subsubsection{The Dataset}\label{the-dataset}}

Where did you find the data? Please include a link to the data source.

The data is found on the website called Kaddle which is an online
platform for data scientists and even learners.
\url{https://www.kaggle.com/datasets/shahzadahmad0402/depression-and-anxiety-data?resource=download}

Who collected the data?

The data was collected by Shahzad Ahmed.

How was the data collected or generated?

The data was collected from undergraduate students at the University of
Lahore where there were 787 participants. The data was generated from
the inspiration of the Beck Depression and Beck Anxiety inventories.

Why was the data collected?

The data was collected ``to evaluate different machine learning methods
and to compare the different machine learnings classification
approaches.''

How many observations (rows) are in your data?

There are a total of 783 observations in the data.

How many features (columns) are in the data?

There are a total of 19 features in the data.

What, if any, ethical questions or questions of power do you need to
consider when working with this data?\\
What are possible limitations or problems with this data? (at least 200
words)

Possible limitations for approaching this data will be due to lack of
explanations and the context behind it. It is often complicated to
measure someone's mental health and usually it is associated with
personal backgrounds and possibly from their childhood. Considering this
fact, measuring someone's mental health within 19 features might not
yield an ideal outcome for this project. Furthermore, some of the
features have only two options to represent the status of a student,
which are `TRUE' and `FALSE.' Therefore, these might also function as
two extreme classifications representing an individual's status.

Another possible limitation can be due to the gender classification of
its data. There are only two categories to represent the identity of a
student, which is excluding the possible factors that could function as
a crucial factor to explain the mental health of a student. Our chosen
data does not provide a holistic view of each student's life. We have to
be very careful when dealing with data that doesn't share a
characteristic of qualitative data. Lastly, although the data shows a
large number of students with high levels of anxiety-severity, only a
small part of the data shows they are diagnosed as depressed or getting
treatment for it.

\hypertarget{implications}{%
\subsubsection{Implications}\label{implications}}

Assuming you answer your research questions, briefly describe the
expected or possible implications for technologists, designers, and
policymakers. (at least 150 words)

\begin{enumerate}
\def\labelenumi{\arabic{enumi})}
\tightlist
\item
  If age does impact mental health, then what?
\item
  If BMI does impact mental health, then what?
\item
  If gender does impact mental health, then what?
\end{enumerate}

Determining the impacts of gender on a student's mental health not only
helps us get an understanding about mental health in general, but also
helps us research in many other areas. For example, biologists can study
what causes these mental health differences between genders. Is there a
chemical in the brain that is more present in a male body or females
body that contributes to mental health issues? Or, is there a lack of a
chemical in a certain sexes body that influences the presence of
depression and mental health? Researchers can help us answer these
questions and get a better grasp of what attributes to these gender
differences in mental health issues.

If there are correlations between college students' mental health issues
and gender, age, and class standing, policy makers can use that data to
push mental health legislation that can alleviate the struggles college
students go through. This could be foundation to help the students who
suffer from the mental health and guide students' to have better mental
health.

\hypertarget{limitations-challenges}{%
\subsubsection{Limitations \& Challenges}\label{limitations-challenges}}

What challenges or limitations might you need to address with your
project idea more broadly? Briefly discuss. (at least 150 words)

One broad challenge we might need to address in regard to gender
affecting a students mental health is the dismantling of gender
stereotypes and gender normative behavior. Because of societies' views
and the environment it fosters when discussing gender, it can be
damaging to those who fit, or fail to fit, in those gender binary
ideologies. Folks who identify as a man may face conflict with other men
in their environment when they don't partake in traditional masculine
activities. This goes the same for folks who do identify as a woman. If
they fail to conform to what society deems as feminine, they may be
outcasted or viewed differently. These gender stereotypes are only
damaging the mental healths of students who are struggling to fit into
these boxes.

One limitation policy makers might run into is successfully implementing
systems that will ultimately aid and support folks struggling with
mental health problems. Because this topic is so polarizing and lacks
bipartisanship, it can be hard to convince policymakers of one party to
come to an agreement regarding mental health care.

Another limitation we should be aware of is the faults that come with
using BMI as a gauge to determine a student's health status. Because BMI
only takes into consideration weight and height, it exposes us to a big
margin of error. Disregarding the muscle mass of an individual can
greatly skew and misinterpret the BMI of an individual.

Another important challenge is the classifications of sex and its
intricacies. The data set only looks at males and females, and
disregards college students who are intersex. It forces those that are
unsure into a gender binary. An additional challenge we need to look at
is deciphering whether a college student's actual sex influenced their
mental health, or if their environment and attitudes towards their sex
is to blame. Because of societies' views and the environment it fosters
when discussing a person's sex, it can be damaging to those who fit, or
fail to fit, in those gender binary ideologies. Folks who identify as a
man may face conflict with other men in their environment when they
don't partake in traditional masculine activities. This goes the same
for folks who do identify as a woman. If they fail to conform to what
society deems as feminine, they may be outcasted or viewed differently.
These gender stereotypes are only damaging the mental health of students
who are struggling to fit into these boxes.

\hypertarget{summary-information}{%
\subsubsection{Summary Information}\label{summary-information}}

All of the summary values are calculated based on the ages between 18
years old to 24 years old, which are the average ages for students. The
calculation shows which gender felt more anxiety using the column,
``anxiousness''. In general, there are more female students who felt
anxiety compared to the male students. In addition, it is found that at
the age of 18, the number of committing a suicide was more than two
times higher than the other ages. At the age of 18, there was 27 suicide
happen. The second highest of the number of suicide is 12 suicide at the
age of 20. Then, the ages of 18, 21, 22, 23, and 24 follow with the
numbers 10, 10, 7, 1, and 1. Usually, the age of 18 is the senior year
in high school. Although the reasons why the students committed suicide
are not given, the number, 18 years old, implies that they are stressed,
depressed, and anxious about going into a new environment such as a
college, work, or etc. This can be useful to prevent students from
committing a suicide. Another value is the number of students who
experienced depression which is found as 209 out of 783 students, which
is approximately 27 percent. The 27 percent means that there is almost
on student who has experienced depression in 3 students. Ironically, the
number of students who have been diagnosed for depression is very low
with the number of 65. In comparison to the number of students diagnosed
for depression, the number of students who committed suicide is too high
with the number of 68. This shows that the students avoid revealing
their mental health status. The values are really important because they
reflect the students' mental health, when to care take of them more
attentively, and how the students' view mental illness. This will help
us to teach students' that the mental illness is not something you have
to hide, but something that you can be treated on. The values also help
us develop medical treatment for the next generation for mental health
disorders.

\hypertarget{table}{%
\subsubsection{Table}\label{table}}

Include a table of aggregate information

Describe why you included the table and what information it reveals

\hypertarget{chart-1}{%
\subsubsection{Chart 1}\label{chart-1}}

This chart is a data visualization that shows the correlation between
gender and depression severity, it shows that while men have higher none
to none-minimal depression, women have higher mild to severe depression.
This data visualization is important because it shows how while both
women and men face depression, women have predominantly worse depression

\textless\textless\textless\textless\textless\textless\textless{} HEAD
Include a chart

Describe why you chose this chart and what information it reveals

Here's an example of how to run an R script inside an RMarkdown file:

\includegraphics{index_files/figure-latex/unnamed-chunk-1-1.pdf}

\begin{verbatim}
## 
## Attaching package: 'dplyr'
\end{verbatim}

\begin{verbatim}
## The following objects are masked from 'package:stats':
## 
##     filter, lag
\end{verbatim}

\begin{verbatim}
## The following objects are masked from 'package:base':
## 
##     intersect, setdiff, setequal, union
\end{verbatim}

\includegraphics{index_files/figure-latex/unnamed-chunk-2-1.pdf}

\hypertarget{chart-2}{%
\subsubsection{Chart 2}\label{chart-2}}

\textless\textless\textless\textless\textless\textless\textless{} HEAD
Include a chart

Describe why you chose this chart and what information it reveals

\includegraphics{index_files/figure-latex/unnamed-chunk-3-1.pdf}

======= This chart is a data visualization that shows the correlation
between age and depression severity it is important to notice that
depression often peaks between age 19-22, this is important because it
can help destigmatize that those with depression in their early twenties
are alone, or you that should be embarrassed to face depression at such
a young age.
\includegraphics{index_files/figure-latex/unnamed-chunk-4-1.pdf}
\textgreater\textgreater\textgreater\textgreater\textgreater\textgreater\textgreater{}
1316f386c3c5b97405a2209b437bd8c3f3ead1ab \#\#\# Chart 3

This chart is a data visualization that shows the correlation between
class standing and depression severity. We wanted to see if certain
college years made a students depression more severe. Since coursework
tends to get harder the further you are into college, does the severity
of depression get worse too? This graph aims to explore and explain this
question.

\textless\textless\textless\textless\textless\textless\textless{} HEAD
Describe why you chose this chart and what information it reveals

\hypertarget{section}{%
\section[]{\texorpdfstring{\protect\includegraphics{index_files/figure-latex/unnamed-chunk-5-1.pdf}}{}}\label{section}}

\includegraphics{index_files/figure-latex/unnamed-chunk-6-1.pdf}
\textgreater\textgreater\textgreater\textgreater\textgreater\textgreater\textgreater{}
1316f386c3c5b97405a2209b437bd8c3f3ead1ab

\end{document}
